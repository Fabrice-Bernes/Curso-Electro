%! TEX root = *master.tex

% chktex-file 2          Don't suggest (a+b)^2 --> {(a+b)}^2
% chktex-file 3          Don't suggest (a+b)^2 --> {(a+b)}^2
\chapter{Clase 1}
\section{Fuerza eléctrica y desplazamiento eléctrico}
La fuerza eléctrica sobre $q_1$, producida por $q_2$, es:
\begin{equation}
    \vect{F}_{2 \rightarrow 1} =
    k \,\, q_1 q_2
    \frac{
        \vect{r}_1 - \vect{r}_2
    }{
        \norma{\vect{r}_1 - \vect{r}_2}^3
    }
\end{equation}

\vspace{-50pt}
\begin{figure}[H]
    \begin{minipage}[c]{0.50\textwidth}
        \centering
        \begin{luadraw}{name=2_cargas,exec=false}
            require "lua.dos_cargas_puntuales" -- % chktex 18 chktex 8
        \end{luadraw}
    \end{minipage}
    \hfill
    \begin{minipage}[c]{0.40\textwidth}
        \caption{
            Dos cargas puntuales $q_1$ y $q_2$, con sus respectivos vectores
            de posición respecto a un sistema de coordenadas cartesiano.
        }%
    \end{minipage}
\end{figure}

\vspace{1em}\hrule\vspace{1em}

Otro escenario posible, es el de una carga puntual que experimenta una fuerza
eléctrica en presencia de un campo externo:
\begin{equation}\label{eq:F=qE}
    \vect{F}(\vect{r}) = q \vect{E}(\vect{r})
\end{equation}

\vspace{-50pt}
\begin{figure}[H]
    \begin{minipage}[c]{0.50\textwidth}
        \centering
        \begin{luadraw}{name=carga_en_campo,exec=false}
            require "lua.carga_en_campo" -- % chktex 18 chktex 8
        \end{luadraw}
    \end{minipage}
    \hfill
    \begin{minipage}[c]{0.40\textwidth}
        \caption{
            Carga en un campo $\vect{E}$ de la forma
            $\vect{E}(\vect{r}) = 0\hatvect{i} + E\hatvect{j} + 0\hatvect{k}$
        }%
    \end{minipage}
\end{figure}

\vspace{1em}\hrule\vspace{1em}

El campo que genera una carga eléctrica, se escribe como:
\begin{equation}\label{campo_E}
    \vect{E}(\vect{r}) =
    k \, q_1
    \frac{\vect{r} - \vect{r}_1}{ \norma{ \vect{r} - \vect{r}_1 }^3 }
\end{equation}

Si se tienen muchas cargas puntuales, aplica el principio de superposición.
Esto es, el campo total es igual a la suma de los campos que cada partícula
produce individualmente, según \eqref{campo_E}:
\begin{equation}
    \vect{E}(\vect{r}) =
    k \sum_{i=1}^N
      q_i \frac{\vect{r} - \vect{r}_i}{ \norma{ \vect{r} - \vect{r}_i }^3 }
\end{equation}

Y en el límite $N\rightarrow\infty \quad q_i \rightarrow 0$:
\begin{equation}\label{eq:campo_distr_continua}
    \vect{E}(\vect{r}) =
    k \int
      \rho(\vect{r}')
      \frac{\vect{r} - \vect{r}'}{ \norma{ \vect{r} - \vect{r}' }^3 }
      \ds{\vect{r}'}^3
\end{equation}

\vspace{-50pt}
\begin{figure}[H]
    \begin{minipage}[c]{0.50\textwidth}
        \centering
        \begin{luadraw}{name=distribucion_de_carga,exec=false}
            require "lua.distribucion_de_cargas" -- % chktex 18 chktex 8
        \end{luadraw}
    \end{minipage}
    \hfill
    \begin{minipage}[c]{0.40\textwidth}
        \caption{
            Valor del campo $\vect{E}$ en el punto $\vect{r}$, producido por
            la distribución continua de carga, con densidad $\rho(\vect{r}')$.
        }%
    \end{minipage}
\end{figure}

\section{La Ley de Gauss}
El teorema de la divergencia es:
\begin{equation}\label{eq:teorema_de_la_divergencia}
    \oint_S \vect{A} \cdot \hatvect{n} \ds{s}
    =
    \int_V \nabla \cdot \vect{A} \ds{v}
\end{equation}

Donde $\vect{A}$ es un campo vectorial definido en el volumen $V$, y
$S$ es la superficie que delimita a $V$. Los vectores $\hatvect{n}$ son
las normales a cada punto en $S$, y la integral sobre $S$ se denomina
\textit{flujo de $\vect{A}$ a través de $S$}.

La ecuación $\nabla \cdot \vect{E} = 4 \pi k \rho$,
o en forma integral,
$
    \int_S \vect{E} \cdot \hatvect{n} \ds{s} =
    \int_V \nabla \cdot \vect{E} \ds{v} =
    4\pi k \int_V \rho(\vect{r}) \ds{\vect{r}}^3
    = 4\pi k Q_V
$,
es consecuencia de \eqref{eq:teorema_de_la_divergencia}.
%
Para probar esto, se puede sustituir
la definición \eqref{eq:campo_distr_continua} en
\eqref{eq:teorema_de_la_divergencia}, con $\vect{A} = \vect{E}$:
\begin{equation}\label{eq:ley_de_gauss_int}
    \int_S \eqnmarkbox[gray]{n1}{\ds{\vect{r}}^2} % annotation
      (\vect{E} \cdot \eqnmarkbox[gray]{n4}{\hatvect{n}}) % annotation
    = k\int_S \ds{\vect{r}}^2
      \Biggl(
            \int_V \eqnmarkbox[gray]{n2}{\ds{\vect{r}'}^3} % annotation
            \Bigl(
              \rho(\vect{r}')
              \frac{\vect{r} - \vect{r}'}{ \norma{ \vect{r} - \vect{r}' }^3 }
              \cdot \eqnmarkbox[gray]{n3}{\hatvect{n}} % annotation
            \Bigr)
      \Biggr)
\end{equation}
\annotate[yshift=-1em]{below,left}{n1}{$\ds{s}$}
\annotate[yshift=-1em]{below,left}{n2}{$\ds{v}$}
\annotate[yshift=-1em]{below,left}{n3}{$\hatvect{n}(\hatvect{r})$}
\annotate[yshift=0.5em]{above,right}{n1,n4}{$\ds{\vect{s}}$}

Las integrales de superficie y de volumen se pueden evaluar en cualquier
orden, y la densidad se evalúa en el dominio de la integral de volumen.
Luego, se puede escribir:
\begin{equation}
    \int_S \ds{\vect{r}}^2
        \int_V \ds{\vect{r}'}^3
          \rho(\vect{r}')
          \frac{\vect{r} - \vect{r}'}{ \norma{ \vect{r} - \vect{r}' }^3 }
      \cdot \hatvect{n}
    = \int_V \ds{\vect{r}'}^3 \rho(\vect{r}')
        \int_S \ds{\vect{r}}^2
          \frac{\vect{r} - \vect{r}'}{ \norma{ \vect{r} - \vect{r}' }^3 }
    \cdot \hatvect{n}
    = \int_V \ds{\vect{r}'}^3 \rho(\vect{r}') F(\vect{r}')
\end{equation}

La integral de superficie, vista como una función de $\vect{r}'$,
es la de el diferencial de ángulo sólido que subtiende $S$ sobre un observador
en $\vect{r}$
(ver sección 4\-1 de ``Campos electromagnéticos'', de Roald K.\ Wangsness).      %TODO bibtex
Para notarlo, reescribimos:

\vspace{2em}
\begin{equation}
    \int_S \ds{\vect{r}}^2
          \frac{\vect{r} - \vect{r}'}{ \norma{ \vect{r} - \vect{r}' }^3 }
    \cdot \hatvect{n}
    = \int_S \ds{\vect{r}}^2
      \frac{
          \eqnmarkbox[gray]{n1}{\vect{R}} % annotation
        }{
            \norma{\vect{R}}^3
        }
      \cdot \hatvect{n}
    = \int_S
      \frac{
          \eqnmarkbox[gray]{n2}{\hatvect{R}} % annotation
          \cdot
          \hatvect{n}
        }{
            \norma{\vect{R}}^2
        }
      \ds{\vect{r}}^2
    = \int_S
      \frac{\cos(\theta)}{ \norma{\vect{R}}^2 }
      \ds{\vect{r}}^2
    = \int_S \ds{\Omega}
\end{equation}
\annotate[yshift=2em]{above,right}{n1}{
    Pos.\ respecto de $\vect{r}$, del
    diferencial de superf.\ con normal $\hatvect{n}$.
}
\annotate[yshift=0.5em]{above,right}{n2}{$\vect{R}\norma{R}^{-1}$}

Luego, el ángulo subtendido por el diferencial en $S$, sería el mismo que el
que subtiende el de un una esfera de radio $a$, centrada en $\vect{r}$,
como se muestra en la figura.                                                   %TODO figura
\begin{equation}
    \ds{\Omega} = \frac{ \ds{s}^2 }{ \norma{ \vect{R} }^2 }
    = \frac{\ds{s'}^2}{a^2}
\end{equation}

Si $\vect{r}$ se encuentra dentro de $S$, entonces todos los elementos de
superficie tendrán normales que apuntan radialmente hacia afuera de la esfera
de radio $a$, y la integral toma el valor:
\begin{equation}
    \int_S \ds{\Omega}
    = \int_S \frac{1}{a^2} \ds{s'}^2
    = \frac{1}{a^2} \int_S \ds{s'}^2
    = \frac{1}{a^2} 4 \pi a^2 = 4\pi
    \qquad\qquad
    \text{ ($\vect{r}$ DENTRO DE $S$) }
\end{equation}

Y si $\vect{r}$ se encuentra fuera de $S$, entonces, para cada elemento de
superficie en $S$, cuya normal no sea perpendicular a su respectivo $\vect{R}$,
existirá otro elemento de superficie que subtienda el mismo ángulo sólido,
pero cuya normal apunte en dirección opuesta, como se muestra en la figura.       %TODO figura
Luego:
\begin{equation}
    \int_S \ds{\Omega} = 0
    \qquad\qquad
    \text{ ($\vect{r}$ FUERA DE $S$) }
\end{equation}

Por lo tanto, la integral \eqref{eq:ley_de_gauss_int} es:
\begin{equation}\label{eq:ley_de_gauss_casos}
    \oint_S \vect{E} \cdot \ds{\vect{s}}^2 =
    \begin{cases}
        \begin{aligned}
            &{\displaystyle 4k\pi \int_V \rho \ds{v}^3 = 4k \pi Q_V }
            \qquad\qquad
            &\text{ SI $\vect{r}$ ESTÁ DENTRO DE $S$ }
            \\[1em]
            &{\displaystyle 0 \int_V \rho \ds{v}^3 = 0 }
            \qquad\qquad % useless here but meh
            &\text{ SI $\vect{r}$ ESTÁ FUERA DE $S$ }
        \end{aligned}
    \end{cases}
\end{equation}

\section{Potencial escalar}
Notando lo siguiente:
\begin{equation}
\begin{aligned}
    \nabla \left( \frac{1}{ \norma{\vect{r} - \vect{r}'} } \right)
    &= \begin{pmatrix}
        {\displaystyle \diffp*{\norma{\vect{r}-\vect{r}'}^{-1}}{x}}
        \\[1em]
        {\displaystyle \diffp*{\norma{\vect{r}-\vect{r}'}^{-1}}{y}}
        \\[1em]
        {\displaystyle \diffp*{\norma{\vect{r}-\vect{r}'}^{-1}}{z}}
    \end{pmatrix}
    = \begin{pmatrix}
        {\displaystyle
            \diffp*{ \frac{1}{ \sqrt{(x-x')^2 + (y-y')^2 + (z-z')^2} } }{x}
        }
        \\[1em]
        {\displaystyle
            \diffp*{ \frac{1}{ \sqrt{(x-x')^2 + (y-y')^2 + (z-z')^2} } }{y}
        }
        \\[1em]
        {\displaystyle
            \diffp*{ \frac{1}{ \sqrt{(x-x')^2 + (y-y')^2 + (z-z')^2} } }{z}
        }
    \end{pmatrix}
    = \cdots
    \\\cdots&
    = \begin{pmatrix}
        {\displaystyle
            - \frac{1}{2}\frac{ 2(x - x') }{
                \Bigl( (x-x')^2 + (y-y')^2 + (z-z')^2 \Bigr)^{3/2}
            }
        }
        \\[1em]
        {\displaystyle
            - \frac{1}{2}\frac{ 2(y - y') }{
                \Bigl( (x-x')^2 + (y-y')^2 + (z-z')^2 \Bigr)^{3/2}
            }
        }
        \\[1em]
        {\displaystyle
            - \frac{1}{2}\frac{ 2(z - z') }{
                \Bigl( (x-x')^2 + (y-y')^2 + (z-z')^2 \Bigr)^{3/2}
            }
        }
    \end{pmatrix}
    = - \begin{pmatrix}
        {\displaystyle \frac{ x - x' }{ \norma{ \vect{r} - \vect{r}' }^3 } }
        \\[1em]
        {\displaystyle \frac{ y - y' }{ \norma{ \vect{r} - \vect{r}' }^3 } }
        \\[1em]
        {\displaystyle \frac{ z - z' }{ \norma{ \vect{r} - \vect{r}' }^3 } }
    \end{pmatrix}
    = - \frac{ \vect{r} - \vect{r}' }{ \norma{\vect{r}-\vect{r}'}^3 }
\end{aligned}
\end{equation}

La definición \eqref{eq:campo_distr_continua} se puede escribir como:
\begin{equation}
    \vect{E}(\vect{r})
    = - k \int \ds{\vect{r}'}^3
      \eqnmarkbox[gray]{n1}{\rho(\vect{r}')} % annotation
      \,
      \eqnmarkbox[gray]{n2}{\nabla} % annotation
      \left( \frac{1}{\norma{\vect{r}-\vect{r}'}} \right)
    = - \int \ds{\vect{r}'}^3
      \nabla \left(
          k \frac{\rho(\vect{r}')}{ \norma{\vect{r}-\vect{r}'} }
      \right)
\end{equation}
\annotate[yshift=0.5em]{above,left}{n1}{Definido en términos de $\vect{r}'$}
\annotate[yshift=-1em]{below,left}{n2}{Definido en términos de $\vect{r}$}

Los operadores $\nabla$ y $\int \ds{v}$ conmutan, y por lo tanto:
\begin{equation}\label{eq:E=-nabla(phi)}
    \vect{E}(\vect{r})
    = - \nabla \left(
          \int \ds{\vect{r}}^3
          \,\,
          k \frac{\rho(\vect{r}')}{ \norma{\vect{r}-\vect{r}'} }
      \right)
    = - \nabla \phi(\vect{r})
\end{equation}
%
donde $\phi$ es es el \textit{potencial escalar}:
\begin{equation}
    \phi(\vect{r})
    = k \int
      \frac{\rho(\vect{r}')}{ \norma{\vect{r}-\vect{r}'} }
      \ds{\vect{r}}^3
\end{equation}

Notar que de $\vect{E} = - \nabla \phi$ se sigue que
$\nabla \times \vect{E} = 0$:
\begin{equation}\label{eq:rotacional_E}
    \nabla \times \vect{E} = - \nabla \times \nabla \phi = 0
\end{equation}

\section{Trabajo eléctrico}
En general, el trabajo realizado por una fuerza $\vect{F}_\text{ext.}$ a
lo largo de una trayectoria es:
\begin{equation}
    W_{a \rightarrow b} = \int_a^b \vect{F}_\text{ext.} \cdot \ds{\vect{\ell}}
\end{equation}

Entonces, en presencia de un campo $E$, el trabajo que debe realizar una fuerza
externa para llevar a una carga $q$ en equilibrio\footnote{
    Es decir que en todo momento se cumple
    $ \vect{F}_\text{ext.} + \vect{F}_\text{elec.} = 0 $.
    Es decir:
    $ \vect{F}_\text{ext.} = - \vect{F}_\text{elec.} $
}, desde $A$ hasta $B$, es,
de acuerdo con \eqref{eq:F=qE} y \eqref{eq:E=-nabla(phi)}:
\begin{equation}
\begin{aligned}
    W_{a \rightarrow b}
    &= \int_A^B \vect{F}_\text{ext.} \cdot \ds{\vect{\ell}}
    = - \int_A^B \vect{F}_\text{elec.} \cdot \ds{\vect{\ell}}
    = - q \int_A^B \vect{E} \cdot \ds{\vect{\ell}}
    = q \int_A^B \nabla \phi \cdot \ds{\vect{\ell}}
    \\
    &= q \int_A^B \left(
          \diffp{\phi}{x}\hatvect{i}
        + \diffp{\phi}{y}\hatvect{j}
        + \diffp{\phi}{z}\hatvect{k}
      \right)
      \cdot
      \left( \ds{x}\hatvect{i} + \ds{y}\hatvect{j} + \ds{z}\hatvect{k} \right)
    \\
    &= q \int_A^B
    \eqnmarkbox[gray]{n1}{    % annotation
          \left(
              \diffp{\phi}{x}\ds{x}
            + \diffp{\phi}{y}\ds{y}
            + \diffp{\phi}{z}\ds{z}
          \right)
    }
    = q \int_A^B \ds{\phi}
    = q \Bigl( \phi(B) - \phi(A) \Bigr)
\end{aligned}
\end{equation}
\annotate[yshift=-0.5em]{below,left}{n1}{
    $\displaystyle\sum_i \diffp{\phi}{x_i}\ds{x_i} = \ds{\phi}(x,y,z)$
}
\vspace{1em}

Para el caso en el que $B$ y $A$ coinciden, se tiene
$W_{A \rightarrow A} = q[ \phi(A) - \phi(A) ] = 0$.
Es decir, el trabajo sobre trayectorias cerradas es nulo. La fuerza eléctrica
es conservativa, como se esperaba de $\nabla \times \vect{E} = 0$:
\begin{equation}
    - q \oint_C \vect{E} \cdot \ds{\vect{\ell}}
    = - q \int_S \Bigl(\nabla \times \vect{E}\Bigr) \cdot \ds{s}^2
    = 0
    \implies \oint \vect{F}_\text{elec.} \cdot \ds{\ell} = 0
\end{equation}

Donde $C$ $S$ son la trayectoria cerrada, y la superficie delimitada por dicha
trayectoria, respectivamente. Ver figura. %TODO FIGURA

\section{
    Distribuciones superficiales de carga y las
    discontinuidades de \texorpdfstring{$\vect{E}$}{E}
}
Dada una distribución superficial de carga, con densidad $\sigma(\vect{r})$,
entonces la divergencia del campo que ésta produce, satisface
$\nabla \cdot \vect{E} = 4\pi k \sigma(\vect{r})$,
que en forma integral, es, de acuerdo con \eqref{eq:ley_de_gauss_casos}:
\begin{equation}\label{eq:gauss_cilindro}
    \int_V \Bigl( \nabla \cdot \vect{E} \Bigr) \ds{v}^3
    = 4\pi \int_S \sigma(\vect{r}) \ds{s}^2
\end{equation}

Donde $V$ es un volumen que contiene por completo
a la distribución de cargas, y $S$ su superficie.
En la figura \ref{fig:cilindro_sigma} se muestra la superficie que delimita
a un cilindro de altura $h$, que cumple las características de $V$.

\vspace{-20pt}
\begin{figure}[H]
    \begin{minipage}[c]{0.50\textwidth}
        \centering
        \begin{luadraw}{name=cilindro_sigma,exec=false}
            require "lua.cilindro" -- % chktex 18 chktex 8
        \end{luadraw}
    \end{minipage}
    \hfill
    \begin{minipage}[c]{0.45\textwidth}
        \caption{
            El cilindro, cuyas tapas están orientadas según las normales
            $\hatvect{n}_1$ y $\hatvect{n}_2 = - \hatvect{n}_1$, contiene
            por completo a la distribución de carga superficial, de densidad
            $\sigma(\vect{r})$.
            \\[1em]
            Los vectores $\vect{E}_1$ y $\vect{E}_2$ son los valores de
            $\vect{E}$ \textit{justo arriba} y \textit{justo abajo} del punto
            $(0,0,0)$, pero por claridad, se acomodó a $\vect{E}_1$ para que
            su punta esté sobre $(0,0,0)$.
        }%
        \label{fig:cilindro_sigma}
    \end{minipage}
\end{figure}

Por otro lado, si se aplica el teorema de la divergencia
\eqref{eq:teorema_de_la_divergencia} y se procede por evaluación directa,
en el límite $h \rightarrow 0$:
\vspace{1em}
\begin{equation}
    \int_V \left( \nabla \cdot \vect{E} \right) \ds{v}
    % = \oint \vect{E} \cdot \hatvect{n} \ds{a}
    = \oint \vect{E} \cdot \ds{\vect{a}}
    = \int_\text{Tapa 1}
      \negphantom{\text{Tapa}}
      % \left( \vect{E}(\vect{r}_1) \cdot \hatvect{n}_1 \right) \ds{s_1}
      \vect{E}(\vect{r}_1) \cdot \ds{\vect{s}_1}
    + \int_\text{Tapa 2}
      \negphantom{\text{Tapa}}
      % \left( \vect{E}(\vect{r}_2) \cdot \hatvect{n}_2 \right) \ds{s_2}
       \vect{E}(\vect{r}_2) \cdot \ds{\vect{s}_2}
    + \int_\text{Pared}
      \negphantom{\text{Par}}
      % \left( \vect{E}(\vect{r}_3) \cdot \hatvect{n}_3(\hatvect{\rho}) \right)
      % \ds{s_3}
      \left(
        \vect{E}(\vect{r}_3)
        \cdot
        \eqnmarkbox[gray]{n1}{\hatvect{\rho}} % annotation
      \right)
      \eqnmarkbox[gray]{n2}{\ds{s_3}} % annotation
\end{equation}
\annotate[yshift=1em]{above,left}{n1,n2}{
    $\ds{\vect{s}_3} = \hatvect{n}_3 \ds{s_3} = \vect{\rho} \ds{s_3}$
}

En el límite $h \rightarrow 0$, el flujo de $\vect{E}$ sobre la
pared del cilindro desaparece:
\begin{equation}\label{eq:flujo_limite}
    \biggl[
        \int_V \left( \nabla \cdot \vect{E} \right) \ds{v}
    \biggr]_{ h \rightarrow 0 }
    = \int_\text{Tapa 1}
      \negphantom{\text{Tapa}}
      % \left( \vect{E}(\vect{r}_1) \cdot \hatvect{n}_1 \right) \ds{s_1}
      \vect{E}(\vect{r}_1) \cdot \ds{\vect{s}_1}
    + \int_\text{Tapa 2}
      \negphantom{\text{Tapa}}
      % \left( \vect{E}(\vect{r}_2) \cdot \hatvect{n}_2 \right) \ds{s_2}
      \vect{E}(\vect{r}_2) \cdot \ds{\vect{s}_2}
\end{equation}

Además, la única diferencia entre las tapas
del cilindro, es que sus normales son opuestas, y por lo tanto:
\vspace{1em}
\begin{equation}\label{normales_opuestas}
    \ds{\vect{s}_1}
    = \hatvect{n}_1 \ds{s_1}
    = \eqnmarkbox[gray]{n1}{\hatvect{n}_1} % annotation
      \,
      \eqnmarkbox[gray]{d1}{\rho \ds{\rho} \ds{ \varphi }} % annotation
    = \eqnmarkbox[gray]{n2}{- \hatvect{n}_2} % annotation
      \rho \ds{\rho} \ds{ \phi }
    = - \hatvect{n}_2 \ds{s_2}
    = - \ds{\vect{s}_{2}}
\end{equation}
\annotate[yshift=1em]{above,left}{n1,n2}{
    $\hatvect{n}_1 = - \hatvect{n}_2 \equiv - \hatvect{n}$
}
\annotate[yshift=-0.5em]{below,right}{d1}{
    $\ds{s_1} = \rho \ds{\rho} \ds{\phi} = \ds{s_2} \equiv \ds{s}$
}

Entonces \eqref{eq:flujo_limite} es:
\begin{equation}\label{eq:flujo_limite_2}
    \biggl[
        \int_V \left( \nabla \cdot \vect{E} \right) \ds{v}
    \biggr]_{ h \rightarrow 0 }
    = \int_S
      \left(
          \vect{E}_1 \cdot \hatvect{n}_1
        + \vect{E}_2 \cdot \hatvect{n}_2
      \right) \ds{s}
    = \int_S
      \left(
          \vect{E}_1 \cdot \hatvect{n}
        - \vect{E}_2 \cdot \hatvect{n}
      \right) \ds{s}
\end{equation}
%
donde $\vect{E}_1$ y $\vect{E}_2$ son los valores del campo
\textit{justo por debajo} y \textit{justo por encima} del plano que contiene
a la distribución de carga superficial, respectivamente.

\textbf{Nota}:
En la figura \ref{fig:cilindro_sigma}, se le llama
de manera incorrecta (para mantener la figura legible),
$\vect{E}_1$ y $\vect{E}_2$ a lo que en realidad serían sólo dos valores
particulares de las funciones que aparecen en
la ecuación \eqref{eq:flujo_limite_2}. Lo correcto sería etiquetar a
los vectores de la figura como $\vect{E}_1(0,0,0)$ y $\vect{E}_2(0,0,0)$.

Luego, comparando \eqref{eq:flujo_limite_2} con \eqref{eq:gauss_cilindro},
la cual es válida para el cilindro de altura límite que se está considerando,
se obtiene la relación entre las componentes perpendiculares a la distribución
de carga:
\begin{equation}
    \int_S
      \left(
            \eqnmarkbox[gray]{n1}{\vect{E}_{1\perp}} % annotation
          - \eqnmarkbox[gray]{n2}{\vect{E}_{2\perp}} % annotation
      \right) \ds{s}
    = \int_S 4\pi \sigma \ds{s}
    \implies
    \vect{E}_{1\perp} - \vect{E}_{2\perp} = 4\pi \sigma
    \implies
    \vect{E}_{1\perp} = \vect{E}_{2\perp} + 4\pi \sigma
\end{equation}
\annotate[yshift=-0.5em]{below,left}{n1}{$\vect{E}_1 \cdot \hatvect{n}$}
\annotate[yshift=-0.5em]{below,left}{n2}{$\vect{E}_2 \cdot \hatvect{n}$}

Para comparar las componentes tangenciales del campo, se puede construir
una superficie que atraviese al plano sobre el que se define a $\sigma$
(ver figura \ref{fig:cilindro_sigma_2}),
y usar y el teorema de Stokes para la evaluar la integral de
$\nabla \times \vect{E}$ como una integral de línea:

\begin{figure}[H]
    \begin{minipage}[c]{0.50\textwidth}
        \centering
        \begin{luadraw}{name=cilindro_sigma_2,exec=false}
            require "lua.cilindro_2" -- % chktex 18 chktex 8
        \end{luadraw}
    \end{minipage}
    \hfill
    \begin{minipage}[c]{0.45\textwidth}
        \caption{
            Por el Teorema de Stokes, la integral de línea de $\vect{E}$ sobre
            la curva cerrada $ABCDA$, es igual a la integral de superficie
            de $\nabla \times \vect{E}$. Además, en el límite $h \rightarrow 0$,
            los segmentos $AB$ y $CD$ no contribuyen, y además el integrando
            serán las componentes paralelas a la distribución de carga,
            \textit{justo arriba} y \textit{justo debajo} de la misma.
        }%
        \label{fig:cilindro_sigma_2}
    \end{minipage}
\end{figure}

\begin{equation}
    \int_S \left( \nabla \times \vect{E} \right) \ds{s}^2
    = \int_C E \cdot \ds{\ell}
\end{equation}
%
que en el límite $h \rightarrow 0$, se cumple
$\ds{\vect{\ell}_1} = - \ds{\vect{\ell}_2}$, similar a lo explicado en
\eqref{normales_opuestas}
\begin{equation}
\begin{aligned}
    \biggl[
          \int_S \left( \nabla \times \vect{E} \right) \ds{s}^2
    \biggr]_{h \rightarrow 0}
    &= \int_B^C \vect{E}_1 \cdot \ds{\vect{\ell}}
    + \eqnmarkbox[gray]{n1}{\int_C^B \vect{E}_2 \cdot ( -\ds{\vect{\ell}} )}  % annotation
    = \int_B^C \vect{E}_1 \cdot \ds{\vect{\ell}}
    - \int_C^B \vect{E}_2 \cdot \ds{\vect{\ell}}
    \\[1.5em]&
    = \int_B^C \Bigl( \vect{E}_1 - \vect{E}_2 \Bigr) \cdot \ds{\vect{\ell}}
    = \int_B^C \Bigl(
        \vect{E}_1 \cdot \hatvect{\ell}- \vect{E}_2 \cdot \hatvect{\ell}
      \Bigr) \ds{\ell}
\end{aligned}
\end{equation}
\annotate[yshift=-0.5em]{below,left}{n1}{
    Cuando $h \rightarrow 0$, las trayectorias $-DA$ y $BC$ coinciden.
}

Donde $\vect{E} \cdot \ds{\vect{\ell}}$ es la componente de $\vect{E}$
que es tangencial a la distribución superficial de carga. Además, como
$\nabla \times \vect{E} = 0$ (ver ecuación \eqref{eq:rotacional_E}),
entonces:
\begin{equation}
    \int_S \left( \nabla \times \vect{E} \right) \ds{s}^2 = 0
    = \int_B^C
      \Bigl( \vect{E}_{1\parallel} - \vect{E}_{2\parallel} \Bigr) \ds{\ell}
    \implies
    \vect{E}_{1\parallel} - \vect{E}_{1\parallel} = 0
    \implies
    \vect{E}_{1\parallel} = \vect{E}_{1\parallel}
\end{equation}

En conclusión, al atravesar una distribución superficial de carga, $\vect{E}$
cambia en su componente perpendicular a la misma, pero no en la perpendicular:
\begin{equation}
\begin{aligned}
    &\vect{E}_{1\parallel} = \vect{E}_{2\parallel}
    \qquad\qquad
    &\vect{E}_{1\perp} = \vect{E}_{1\perp} + 4\pi \sigma
\end{aligned}
\end{equation}
