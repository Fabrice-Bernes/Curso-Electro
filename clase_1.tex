%! TEX root = *master.tex

% chktex-file 2          Don't suggest (a+b)^2 --> {(a+b)}^2
% chktex-file 3          Don't suggest (a+b)^2 --> {(a+b)}^2
\chapter{Clase 1}
\section{Fuerza eléctrica y campo eléctrico}
La fuerza eléctrica sobre $q_1$, producida por $q_2$, es:
\begin{equation}
    \vect{F}_{2 \rightarrow 1} =
    k \,\, q_1 q_2
    \frac{
        \vect{r}_1 - \vect{r}_2
    }{
        \norma{\vect{r}_1 - \vect{r}_2}^3
    }
\end{equation}

\vspace{-50pt}
\begin{figure}[H]
    \begin{minipage}[c]{0.50\textwidth}
        \centering
        \begin{luadraw}{name=2_cargas,exec=false}
            require "lua.dos_cargas_puntuales" -- % chktex 18 chktex 8
        \end{luadraw}
    \end{minipage}
    \hfill
    \begin{minipage}[c]{0.40\textwidth}
        \caption{
            Dos cargas puntuales $q_1$ y $q_2$, con sus respectivos vectores
            de posición respecto a un sistema de coordenadas cartesiano.
        }%
    \end{minipage}
\end{figure}

\vspace{1em}\hrule\vspace{1em}

Otro escenario posible, es el de una carga puntual que experimenta una fuerza
eléctrica en presencia de un campo externo:
\begin{equation}
    \vect{F}(\vect{r}) = q \vect{E}(\vect{r})
\end{equation}

\vspace{-50pt}
\begin{figure}[H]
    \begin{minipage}[c]{0.50\textwidth}
        \centering
        \begin{luadraw}{name=carga_en_campo,exec=false}
            require "lua.carga_en_campo" -- % chktex 18 chktex 8
        \end{luadraw}
    \end{minipage}
    \hfill
    \begin{minipage}[c]{0.40\textwidth}
        \caption{
            Carga en un campo eléctrico de la forma
            $\vect{E}(\vect{r}) = 0\hatvect{i} + E\hatvect{j} + 0\hatvect{k}$
        }%
    \end{minipage}
\end{figure}

\vspace{1em}\hrule\vspace{1em}

El campo eléctrico que genera una carga eléctrica, se escribe como:
\begin{equation}\label{eq:campo electrico}
    \vect{E}(\vect{r}) =
    k \, q_1
    \frac{\vect{r} - \vect{r}_1}{ \norma{ \vect{r} - \vect{r}_1 }^3 }
\end{equation}

Si se tienen muchas cargas puntuales, aplica el principio de superposición.
Esto es, el campo total es igual a la suma de los campos que cada partícula
produce individualmente, según \eqref{eq:campo electrico}:
\begin{equation}
    \vect{E}(\vect{r}) =
    k \sum_{i=1}^N
      q_i \frac{\vect{r} - \vect{r}_i}{ \norma{ \vect{r} - \vect{r}_i }^3 }
\end{equation}

Y en el límite $N\rightarrow\infty \quad q_i \rightarrow 0$:
\begin{equation}\label{eq:campo_distr_continua}
    \vect{E}(\vect{r}) =
    k \int
      \rho(\vect{r}')
      \frac{\vect{r} - \vect{r}'}{ \norma{ \vect{r} - \vect{r}' }^3 }
      \ds{\vect{r}'}^3
\end{equation}

\vspace{-50pt}
\begin{figure}[H]
    \begin{minipage}[c]{0.50\textwidth}
        \centering
        \begin{luadraw}{name=distribucion_de_carga,exec=false}
            require "lua.distribucion_de_cargas" -- % chktex 18 chktex 8
        \end{luadraw}
    \end{minipage}
    \hfill
    \begin{minipage}[c]{0.40\textwidth}
        \caption{
            Valor del campo $\vect{E}$ en el punto $\vect{r}$, producido por
            la distribución continua de carga, con densidad $\rho(\vect{r}')$.
        }%
    \end{minipage}
\end{figure}

\section{La Ley de Gauss}
El teorema de la divergencia es:
\begin{equation}\label{eq:teorema_de_la_divergencia}
    \oint_S \vect{A} \cdot \hatvect{n} \ds{s}^2
    =
    \int_V \nabla \cdot \vect{A} \ds{v}^3
\end{equation}

Donde $\vect{A}$ es un campo vectorial definido en el volumen $V$, y
$S$ es la superficie que delimita a $V$. Los vectores $\hatvect{n}$ son
las normales a cada punto en $S$, y la integral sobre $S$ se denomina
\textit{flujo de $\vect{A}$ a través de $S$}.

Con el teorema de la divergencia, la ecuación
$\nabla \cdot \vect{E} = 4 \pi k \rho$ se puede expresar en forma
integral:

\begin{equation}\label{eq:Ley_de_gauss}
    \int_S \vect{E} \cdot \hatvect{n} \ds{s} =
    \int_V \nabla \cdot \vect{E} \ds{v} =
    4\pi \int_V \rho(\vect{r}) \ds{\vect{r}}^3
    = 4\pi Q_V
\end{equation}

Luego, sustituyendo la definición \eqref{eq:campo_distr_continua}
en \eqref{eq:Ley_de_gauss}:
\begin{equation}
\int_S \ds{\vect{r}}^2 (\vect{E} \cdot \hatvect{n}) =
    k\int_S \ds{\vect{r}}^2
        \int_V \ds{\vect{r}'}^3
          \rho(\vect{r}')
          \frac{\vect{r} - \vect{r}'}{ \norma{ \vect{r} - \vect{r}' }^3 }
    \cdot \hatvect{n}
    % = k \int_V \ds{\vect{r}'}^3 \rho(\vect{r}')
    %     \int_S \ds{\vect{r}}^2
    %       \frac{\vect{r} - \vect{r}'}{ \norma{ \vect{r} - \vect{r}' }^3 }
    % \cdot \hatvect{n} =
    % k \int_V \ds{\vect{r}'}^3 \rho(\vect{r}') F(\vect{r}')
\end{equation}
